\documentclass[12pt,a4paper]{report}
\usepackage[utf8]{inputenc}
\usepackage[T1]{fontenc}
\usepackage[french]{babel}
\usepackage{graphicx}
\usepackage{hyperref}
\usepackage{geometry}
\usepackage{titlesec}
\usepackage{fancyhdr}
\usepackage{listings}
\usepackage{xcolor}
\usepackage{float}
\usepackage{tocloft}
\usepackage{tikz} 
\usepackage{eso-pic} 
\usepackage{setspace}

% --- COULEURS PROFESSIONNELLES (Thème Bleu Tech) ---
\definecolor{primaryColor}{RGB}{0, 51, 102}    % Bleu Marine Profond
\definecolor{secondaryColor}{RGB}{0, 119, 190} % Bleu Océan Vif
\definecolor{accentColor}{RGB}{240, 248, 255}  % Bleu Alice (Fond très clair)
\definecolor{grayText}{RGB}{80, 80, 80}

% Configuration de la page
\geometry{
    top=2.5cm,
    bottom=2.5cm,
    left=2.5cm,
    right=2.5cm
}

% Configuration des liens (en couleur maintenant)
\hypersetup{
    colorlinks=true,
    linkcolor=primaryColor,
    filecolor=secondaryColor,      
    urlcolor=secondaryColor,
    pdftitle={Rapport Smart Document Manager},
    pdfpagemode=FullScreen,
}

% Configuration du code
\definecolor{codegreen}{rgb}{0,0.6,0}
\definecolor{codegray}{rgb}{0.5,0.5,0.5}
\definecolor{backcolour}{rgb}{0.96,0.96,0.98}

\lstdefinestyle{mystyle}{
    backgroundcolor=\color{backcolour},   
    commentstyle=\color{codegreen},
    keywordstyle=\color{secondaryColor},
    numberstyle=\tiny\color{codegray},
    stringstyle=\color{primaryColor},
    basicstyle=\ttfamily\footnotesize,
    breakatwhitespace=false,         
    breaklines=true,                 
    captionpos=b,                    
    keepspaces=true,                 
    numbers=left,                    
    numbersep=5pt,                  
    showspaces=false,                
    showstringspaces=false,
    showtabs=false,                  
    tabsize=2,
    frame=single,
    rulecolor=\color{secondaryColor}
}

\lstset{style=mystyle}

% --- DESIGN DES TITRES ET SECTIONS EN COULEUR ---

% Chapitre : Grand numéro coloré + Titre coloré
\titleformat{\chapter}[display]
  {\normalfont\huge\bfseries\color{primaryColor}}
  {\titlerule[2pt]\vspace{0.8cm}\centering\chaptertitlename\ \thechapter\vspace{0.8cm}\titlerule[1pt]}
  {20pt}
  {\Huge\centering}
% Espacement
\titlespacing*{\chapter}{0pt}{-30pt}{30pt}

% Section : Couleur secondaire avec une ligne de soulignement
\titleformat{\section}
  {\normalfont\Large\bfseries\color{secondaryColor}}
  {\thesection}{1em}{}

% Sous-section : Couleur primaire
\titleformat{\subsection}
  {\normalfont\large\bfseries\color{primaryColor}}
  {\thesubsection}{1em}{}

% --- EN-TÊTE ET PIED DE PAGE COLORÉS ---
\pagestyle{fancy}
\fancyhf{}
\fancyhead[L]{\textcolor{secondaryColor}{\textit{Smart Document Manager}}}
\fancyhead[R]{\textcolor{grayText}{\leftmark}}
\fancyfoot[C]{\thepage}

% Couleur de la ligne de séparation d'en-tête
\renewcommand{\headrulewidth}{1pt}
\renewcommand{\headrule}{\hbox to\headwidth{\color{secondaryColor}\leaders\hrule height \headrulewidth\hfill}}

\begin{document}

% --- PAGE DE GARDE DESIGN & COLORÉE ---
\begin{titlepage}
    \begin{tikzpicture}[remember picture,overlay]
        % Forme géométrique colorée en haut à droite
        \fill[primaryColor] (current page.north east) -- +(-8cm,0) -- +(0,-8cm) -- cycle;
        \fill[secondaryColor] (current page.north east) -- +(-4cm,0) -- +(0,-4cm) -- cycle;
        
        % Bande latérale gauche
        \fill[primaryColor] (current page.north west) rectangle ([xshift=2cm]current page.south west);
        \node[rotate=90, text=white, anchor=center, font=\Huge\bfseries] at ([xshift=1cm, yshift=0cm]current page.west) {RAPPORT DE PROJET};
    \end{tikzpicture}
    
    \centering
    \vspace*{2cm}
    
    % En-tête école
    {\Huge \textbf{\textcolor{primaryColor}{ENSET}}}\\[0.5cm]
    {\Large \textcolor{grayText}{École Normale Supérieure de l'Enseignement Technique}}\\[2.5cm]

    % Titre du projet dans une boite stylisée
    \begin{tikzpicture}
        \node[
            draw=secondaryColor, 
            line width=2pt, 
            fill=accentColor, 
            rounded corners=15pt, 
            inner sep=20pt, 
            text width=12cm, 
            align=center, 
            drop shadow
        ] {
            {\huge \bfseries \textcolor{primaryColor}{Smart Document Manager}}\\[0.8cm]
            {\Large \textcolor{secondaryColor}{Système de Gestion Intelligente des Documents avec OCR et IA}}
        };
    \end{tikzpicture}
    
    \vspace*{1.5cm}
    \textsc{\Large \textcolor{grayText}{Filière : Ingénierie Informatique}}\\[3cm]

    % Auteurs et Encadrants
    \begin{minipage}{0.45\textwidth}
        \begin{flushleft} \large
            \textcolor{grayText}{\textit{Réalisé par :}}\\
            \vspace{0.2cm}
            \textbf{\textcolor{primaryColor}{Lafkiar Rachid}}
        \end{flushleft}
    \end{minipage}
    \hfill
    \begin{minipage}{0.45\textwidth}
        \begin{flushright} \large
            \textcolor{grayText}{\textit{Encadré par :}}\\
            \vspace{0.2cm}
            \textbf{\textcolor{primaryColor}{[Nom de l'encadrant]}}
        \end{flushright}
    \end{minipage}

    \vfill

    % Année en bas
    {\Large \bfseries \textcolor{secondaryColor}{Année Universitaire : 2024 - 2025}}
    \vspace*{1.5cm}

\end{titlepage}

% --- DÉDICACES ---
\chapter*{Dédicaces}
\addcontentsline{toc}{chapter}{Dédicaces}

\vspace*{3cm}
\begin{center}
    \Large \itshape \color{grayText}
    
    À mes très chers parents, sources de ma vie et de mon inspiration,\\
    pour leur amour inconditionnel, leurs sacrifices et leur soutien indéfectible.\\
    Aucun mot ne saurait exprimer ma gratitude envers vous.
    
    \vspace{1.5cm}
    
    À mes enseignants, qui ont éclairé mon chemin par leur savoir.\\
    
    \vspace{1.5cm}
    
    À mes amis et collègues, pour les moments inoubliables partagés ensemble.
\end{center}
\vspace*{\fill}

% --- REMERCIEMENTS ---
\chapter*{Remerciements}
\addcontentsline{toc}{chapter}{Remerciements}

\setlength{\parindent}{0cm}
\setlength{\parskip}{1em}
\large \color{black}

Je tiens à exprimer ma profonde gratitude à mon encadrant pour sa disponibilité, ses conseils avisés et son accompagnement tout au long de ce projet. Ses directives m'ont permis de surmonter les obstacles et d'atteindre les objectifs fixés.

Mes remerciements vont également à l'ensemble du corps professoral de l'ENSET pour la qualité de la formation et des connaissances transmises.

Enfin, je remercie toutes les personnes qui ont contribué de près ou de loin à l'aboutissement de ce travail.

% --- LISTE DES ABRÉVIATIONS ---
\chapter*{Liste des Abréviations}
\addcontentsline{toc}{chapter}{Liste des Abréviations}

\begin{itemize} \itemsep1em
    \item \textbf{\textcolor{primaryColor}{OCR}} : Optical Character Recognition (Reconnaissance Optique de Caractères)
    \item \textbf{\textcolor{primaryColor}{IA}} : Intelligence Artificielle
    \item \textbf{\textcolor{primaryColor}{API}} : Application Programming Interface
    \item \textbf{\textcolor{primaryColor}{REST}} : Representational State Transfer
    \item \textbf{\textcolor{primaryColor}{URL}} : Uniform Resource Locator
    \item \textbf{\textcolor{primaryColor}{SQL}} : Structured Query Language
    \item \textbf{\textcolor{primaryColor}{TF-IDF}} : Term Frequency-Inverse Document Frequency
    \item \textbf{\textcolor{primaryColor}{UI}} : User Interface (Interface Utilisateur)
    \item \textbf{\textcolor{primaryColor}{UX}} : User Experience (Expérience Utilisateur)
    \item \textbf{\textcolor{primaryColor}{CSV}} : Comma-Separated Values
    \item \textbf{\textcolor{primaryColor}{JSON}} : JavaScript Object Notation
\end{itemize}

\clearpage
% --- SOMMAIRE ---
{
  \hypersetup{linkcolor=black} % Sommaire en noir pour lisibilité
  \tableofcontents
}

% --- CONTENU DU RAPPORT ---

\chapter{Introduction}
La transformation numérique est aujourd'hui une nécessité pour toutes les organisations. Avec l'augmentation exponentielle du volume de documents numériques et numérisés, la gestion efficace de ces informations devient un défi majeur. Le tri manuel, la saisie de données et l'archivage sont des tâches chronophages et sujettes aux erreurs humaines.

C'est dans ce contexte que s'inscrit le projet \textbf{\textcolor{primaryColor}{Smart Document Manager}}. Il s'agit d'une application web complète visant à automatiser le traitement des documents administratifs (factures, CV, contrats, lettres, etc.).

Ce projet combine plusieurs technologies avancées :
\begin{itemize}
    \item La \textbf{reconnaissance optique de caractères (OCR)} pour extraire le texte des images et PDF.
    \item L'\textbf{intelligence artificielle (Machine Learning)} pour classifier automatiquement les documents selon leur contenu.
    \item Une \textbf{interface web moderne} et intuitive pour visualiser et gérer les documents.
\end{itemize}

Ce rapport détaille les différentes étapes de réalisation de ce projet, depuis l'analyse des besoins jusqu'à l'implémentation technique et les résultats obtenus.

\chapter{Cadre Général}

\section{Contexte du Projet}
Ce projet a été réalisé dans le cadre du cursus académique à l'ENSET (École Normale Supérieure de l'Enseignement Technique). Il vise à mettre en pratique les connaissances acquises en développement logiciel, en traitement d'images, et en intelligence artificielle.

\section{Problématique}
Les entreprises et les administrations traitent quotidiennement des centaines de documents. Le processus traditionnel implique :
\begin{enumerate}
    \item La réception du document papier ou numérique.
    \item La lecture manuelle pour identifier le type de document (facture, devis, CV...).
    \item La saisie manuelle des informations clés.
    \item L'archivage dans le dossier correspondant.
\end{enumerate}
Ce processus est lent, coûteux et inefficace. Comment optimiser ce flux de travail grâce aux technologies modernes ?

\section{Objectifs du Projet}
L'objectif principal est de développer une solution capable de :
\begin{itemize}
    \item \textbf{Numériser et uploader} facilement des documents (PDF, Images).
    \item \textbf{Extraire automatiquement} le contenu textuel utilisable via OCR (Tesseract).
    \item \textbf{Classifier} le document dans la bonne catégorie (Facture, CV, Contrat, etc.) sans intervention humaine.
    \item \textbf{Stocker} les documents et leurs métadonnées dans une base de données sécurisée.
    \item \textbf{Visualiser} des statistiques sur les documents traités via un tableau de bord.
\end{itemize}

\chapter{Conception}

\section{Architecture Globale}
Le projet repose sur une architecture client-serveur moderne, séparant clairement le backend (logique métier et données) du frontend (interface utilisateur).

\begin{figure}[H]
    \centering
    % \includegraphics[width=0.8\textwidth]{architecture_diagram.png}
    \fbox{
    \begin{minipage}[c][5cm]{0.8\textwidth}
        \centering \textbf{\textcolor{secondaryColor}{PLACEHOLDER: Diagramme d'Architecture Globale}} \\
        \small (Client Streamlit $\rightleftharpoons$ API FastAPI $\rightleftharpoons$ Services OCR/IA $\rightleftharpoons$ Base de Données PostgreSQL)
    \end{minipage}
    }
    \caption{Architecture du système SmartDocManager}
\end{figure}

\section{Diagrammes UML}
Pour assurer une conception robuste, plusieurs diagrammes UML ont été élaborés.

\subsection{Diagramme de Cas d'Utilisation}
Les principaux acteurs sont l'Utilisateur Visiteur (accès limité) et l'Utilisateur Connecté (accès complet). Les cas d'utilisation incluent : S'inscrire, Se connecter, Uploader un document, Consulter le Dashboard, Exporter les données.

\begin{figure}[H]
    \centering
    % \includegraphics[width=0.8\textwidth]{usecase_diagram.png}
    \fbox{
    \begin{minipage}[c][8cm]{0.8\textwidth}
        \centering \textbf{\textcolor{secondaryColor}{PLACEHOLDER: Diagramme de Cas d'Utilisation}}
    \end{minipage}
    }
    \caption{Diagramme de Cas d'Utilisation}
\end{figure}

\subsection{Diagramme de Classes}
Le modèle de données s'articule autour des entités principales : `User` (Utilisateur) et `Document`. Un utilisateur peut posséder plusieurs documents. Le document contient des attributs comme le nom de fichier, le texte extrait, la catégorie prédite et le score de confiance.

\begin{figure}[H]
    \centering
    % \includegraphics[width=0.8\textwidth]{class_diagram.png}
    \fbox{
    \begin{minipage}[c][8cm]{0.8\textwidth}
        \centering \textbf{\textcolor{secondaryColor}{PLACEHOLDER: Diagramme de Classes}}
    \end{minipage}
    }
    \caption{Diagramme de Classes (Entités backend)}
\end{figure}

\subsection{Diagramme de Séquence}
Le diagramme de séquence illustre les interactions entre l'utilisateur, l'interface, l'API et la base de données lors d'un scénario clé, comme l'upload et l'analyse d'un document.

\begin{figure}[H]
    \centering
    % \includegraphics[width=0.8\textwidth]{sequence_diagram.png}
    \fbox{
    \begin{minipage}[c][8cm]{0.8\textwidth}
        \centering \textbf{\textcolor{secondaryColor}{PLACEHOLDER: Diagramme de Séquence}}
    \end{minipage}
    }
    \caption{Diagramme de Séquence (Scénario d'Upload)}
\end{figure}

\section{Modèle de Machine Learning}
Le cœur intelligent du système repose sur un modèle de classification supervisée.
\begin{itemize}
    \item \textbf{Vectorisation} : Utilisation de TF-IDF (Term Frequency-Inverse Document Frequency) pour transformer le texte brut en vecteurs numériques.
    \item \textbf{Algorithme} : Comparaison entre Naive Bayes, Logistic Regression et SVM. Le modèle final permet de distinguer les catégories : Facture, CV, Contrat, Lettre, Autre.
    \item \textbf{Entraînement} : Le modèle est entraîné sur un corpus de textes annotés.
\end{itemize}

\chapter{Réalisation}

\section{Choix Technologiques}
Le développement a été réalisé avec la stack technique suivante :

\subsection{Backend}
\begin{itemize}
    \item \textbf{Python 3.9+} : Langage principal.
    \item \textbf{FastAPI} : Pour la création d'une API REST haute performance.
    \item \textbf{PostgreSQL} : Base de données relationnelle robuste.
    \item \textbf{SQLAlchemy} : ORM pour l'interaction avec la base de données.
    \item \textbf{Tesseract OCR} : Moteur open-source pour l'extraction de texte.
\end{itemize}

\subsection{Frontend}
\begin{itemize}
    \item \textbf{Streamlit} : Framework Python pour créer rapidement des applications de données interactives.
    \item \textbf{Plotly} : Bibliothèque pour la génération de graphiques dynamiques.
\end{itemize}

\section{Interface Utilisateur}
L'interface a été conçue pour offrir une expérience utilisateur fluide (UX) et un design moderne (UI).

\subsection{Page d'Accueil}
La page d'accueil (\texttt{Accueil.py}) présente l'application et offre un mode "Visiteur" pour tester l'analyse sans inscription.

\begin{figure}[H]
    \centering
    % \includegraphics[width=0.9\textwidth]{screenshot_accueil_1.png}
    \fbox{
    \begin{minipage}[c][6cm]{0.9\textwidth}
        \centering \textbf{\textcolor{secondaryColor}{PLACEHOLDER: Capture Accueil 1 (Haut de page)}}
    \end{minipage}
    }
    
    \vspace{0.5cm}
    
    % \includegraphics[width=0.9\textwidth]{screenshot_accueil_2.png}
    \fbox{
    \begin{minipage}[c][6cm]{0.9\textwidth}
        \centering \textbf{\textcolor{secondaryColor}{PLACEHOLDER: Capture Accueil 2 (Bas de page / Détails)}}
    \end{minipage}
    }
    \caption{Vues de la Page d'Accueil}
\end{figure}

\subsection{Authentification (Login / Register)}
Les pages sécurisées permettent aux utilisateurs de créer un compte et de se connecter pour sauvegarder leurs documents.
\begin{itemize}
    \item \texttt{0\_Login.py} : Formulaire de connexion.
    \item \texttt{1\_Register.py} : Formulaire d'inscription.
\end{itemize}
\begin{figure}[H]
    \centering
    \fbox{
    \begin{minipage}[c][6cm]{0.9\textwidth}
        \centering \textbf{\textcolor{secondaryColor}{PLACEHOLDER: Capture Login}}
    \end{minipage}
    }
    \par\vspace{0.2cm}
    \textbf{Page de Connexion}
    
    \vspace{0.8cm}
    
    \fbox{
    \begin{minipage}[c][6cm]{0.9\textwidth}
        \centering \textbf{\textcolor{secondaryColor}{PLACEHOLDER: Capture Register}}
    \end{minipage}
    }
    \par\vspace{0.2cm}
    \textbf{Page d'Inscription}
    
    \caption{Interfaces d'Authentification}
\end{figure}

\subsection{Upload et Analyse}
La page \texttt{2\_Upload.py} est le cœur de l'application. Elle permet le glisser-déposer de fichiers. Une fois le document envoyé, le backend extrait le texte et le classifie en temps réel.

\begin{figure}[H]
    \centering
    % \includegraphics[width=0.9\textwidth]{screenshot_upload.png}
    \fbox{
    \begin{minipage}[c][8cm]{0.9\textwidth}
        \centering \textbf{\textcolor{secondaryColor}{PLACEHOLDER: Capture de la page d'Upload et Résultats}}
    \end{minipage}
    }
    \caption{Interface d'upload et affichage des résultats de classification}
\end{figure}

\subsection{Gestion des Documents}
La page \texttt{3\_Documents.py} liste tous les fichiers sauvegardés par l'utilisateur avec des options de filtrage et de recherche.

\begin{figure}[H]
    \centering
    % \includegraphics[width=0.9\textwidth]{screenshot_documents.png}
    \fbox{
    \begin{minipage}[c][8cm]{0.9\textwidth}
        \centering \textbf{\textcolor{secondaryColor}{PLACEHOLDER: Capture de la liste des documents}}
    \end{minipage}
    }
    \caption{Gestionnaire de documents}
\end{figure}

\subsection{Tableau de Bord (Dashboard)}
La page \texttt{4\_Dashboard.py} offre une vue d'ensemble analytique. Les graphiques montrent la répartition des documents par catégorie et la confiance moyenne du modèle.

\begin{figure}[H]
    \centering
    % \includegraphics[width=0.9\textwidth]{screenshot_dashboard.png}
    \fbox{
    \begin{minipage}[c][8cm]{0.9\textwidth}
        \centering \textbf{\textcolor{secondaryColor}{PLACEHOLDER: Capture du Dashboard}}
    \end{minipage}
    }
    \caption{Tableau de bord statistique}
\end{figure}

\chapter{Conclusion}
Le projet \textbf{\textcolor{primaryColor}{Smart Document Manager}} a permis de réaliser une application fonctionnelle répondant à la problématique de la gestion documentaire automatisée. L'intégration de l'OCR et du Machine Learning offre un gain de temps considérable pour le tri des documents.

\section{Bilan}
\begin{itemize}
    \item Une application full-stack opérationnelle avec authentification.
    \item Un modèle de classification atteignant de bonnes performances (précision > 85\%).
    \item Une interface utilisateur moderne et responsive.
\end{itemize}

\section{Perspectives}
Pour l'avenir, plusieurs améliorations sont envisageables :
\begin{itemize}
    \item Support de formats de fichiers supplémentaires (Word, Excel).
    \item Amélioration du modèle IA avec plus de données d'entraînement.
    \item Mise en place d'une recherche plein texte dans le contenu des documents.
    \item Déploiement de l'application sur un serveur cloud (Dockerisation).
\end{itemize}

Ce projet a été une excellente opportunité de consolider les compétences en développement Python et en science des données.

\end{document}
